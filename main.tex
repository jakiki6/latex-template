\documentclass[8pt, letterpaper]{article}
\usepackage[ngerman]{babel}
\usepackage{graphicx}
\usepackage[utf8]{inputenc}
\usepackage{gensymb}
\usepackage{blindtext}
\usepackage{geometry}
\usepackage{fancyhdr}
\usepackage{url}
\usepackage{seqsplit}
\usepackage{hyperref}
\usepackage[version=4]{mhchem}

\graphicspath{{images/}}
\setlength{\parindent}{0cm}
\bibliographystyle{gerplain}
\geometry{letterpaper, margin=1in}
\pagestyle{fancy}

\lhead{Jakob Kirsch}
\rhead{\includegraphics[width=3cm]{logo}}

\newcommand{\mpl}[1]{#1 \(\frac{mol}{l}\)}

\begin{document}
\section{Thema}
Gegeben ist ein Nahrungsergänzungsmittel mit Calcium in Tablettenform.
Es soll überprüft werden, ob die Herstellerangabe des Calciumgehalts von 150mg pro Tablette stimmt.

\section{Hypothese}
Aufgrund der deutschen Gesetze ist es definitiv im Interesse des Herstellers, diese einzuhalten und somit keine Falschangaben zu machen.

\section{Sicherheitshinweise}
Aufgrund der verwendeten Chemikalien (HCl, NaOH, Bromthymolblau) sollten entsprechende Schutzkleidung getragen werden.
Bei der Befüllung der Bürette sollte diese auf den Boden gestellt werden. Zudem sollte die Bürette durch den zweiten Glastrichter befüllt werden, wobei dieser leicht angehoben werden sollte, um eventuelles Verschütten vorzubeugen.
Bei der Handhabung der Salzsäure sollten Handschuhe getragen werden.

\section{Material}
\begin{itemize}
    \item 1 Tablette
    \item Mörser
    \item Destilliertes Wasser ~ 100ml
    \item HCl (\mpl{1}) \(\ge\) 15ml
    \item NaOH (\mpl{0.1})
    \item Bromthymolblau pH-Indikator
    \item 1 Bürette
    \item 1 Erlenmeyerkolben
    \item 1 Messkolben
    \item 2 kleine Glastrichter (sollte in Bürette passen)
    \item 2 Vollpipette (10ml und 15ml) + Peleusball
    \item 1 Abfallbehälter \(\ge\) 1l
\end{itemize}

\section{Versuchsbeschreibung}
Sicherheitshinweise lesen!

Die Idee ist es, eine Rücktitration zu machen, da Calciumcarbonat schlecht wasserlöslich ist und man es in 38.5L Wasser lösen müsste, wenn die Herstellerangabe richtig ist.
Daher wird das Calciumcarbonat in einer bestimmten Menge an Salzsäure gelöst, mit welcher es reagiert, um danach die übrige Salzsäure zu titrieren.

Zuerst wird die Tablette gemörsert und mithilfe des ersten Trichters in den Messkolben gegeben. Hinzu werden exakt 15ml der Salzsäure gegeben. Der Kolben wird danach bis exakt 100ml mit destilliertem Wasser gefüllt.

Die Bürette wird mit Natronlauge gespült, um eventuelle Verunreinigungen zu vermeiden. Zudem kann der Erlenmeyerkolben mit destilliertem Wasser gespült werden, da dieser in unserem Fall kontaminiert war und somit die erste Titration um den Faktor 3 verfälscht hat.

Danach wird die Bürette bis zu dem 0ml Strich befüllt. Es werden exakt 10ml der Lösung im Messkolben zum Erlenmeyerkolben hinzugefügt. Hinzu werden 3 Tropfen Bromthymolblau gegeben.
Diese Lösung wird unter die Bürette gestellt. Aus dieser lässt man langsam die Natronlauge tropfen (während man den Erlenmeyerkolben schwenkt), bis die Lösung zu einer blauen Färbung umkippt.
Die benötigte Menge an Natronlauge wird an der Bürette abgelesen. Anschließend wird der Erlenmeyerkolben in den Abfallbehälter entleert und die Bürette wieder bis auf 0ml befüllt.
Diese Titration wird 3 mal durchgeführt, wobei der erste Durchlauf nur grob ist, um die Größenordnung festzustellen.

Am Ende wird der Inhalt des Abfallbehälters in den Abfluss geschüttet, da er zum größten Teil aus Calciumchlorid und Natriumchlorid und Wasser besteht.
Der restliche Inhalt der Bürette kann wiederbenutzt werden oder kann ebenfalls in den Abfluss entsorgt werden.

\section{Beobachtung}
Bei den 3 Durchgängen kamen folgende Werte heraus: 8.6ml, 7.7ml und 8.2ml \\
Erstaunlicherweise war der Umschwung von Orange nach Blau relativ schnell, wobei dieser manchmal auch wieder umgekehrt wurde, wenn man etwas geschwenkt hat. In einem Fall war die Lösung kurzzeitig sogar Grün.

\section{Auswertung}
Aufgrund unserer Verfälschung im ersten Durchgang wurde dieser wiederholt, wodurch alle 3 Werte einer feinen Titration entsprechen.
Somit ist der Mittelwert 8.17ml.

1 Mol Natronlauge neutralisiert 1 Mol Salzsäure:
\begin{center}
    \ce{HCl + NaOH <=> NaCl + H2O}
\end{center}

Da die Salzsäure 10 mal stärker konzentriert ist als die Natronlauge, wissen wir, dass der Salzsäuregehalt der Lösung 0.817ml Salzsäure mit einer Konzentration von \mpl{1} ist.
Somit wissen wir, dass die Lösung also 0.000817 Mol Salzsäure enthält. Da die Lösung 10ml von den 100ml im Messkolben entspricht, enthält dieser also 0.00817 Mol Salzsäure.
Rechnerisch sollte dieser aber 0.015 Mol Salzsäure enthalten, da er mit 15ml Salzsäure mit \mpl{1} befüllt wurde. Daher hat das Calciumcarbonat der Tablette 0.015 Mol - 0.00817 Mol, also 0.00683 Mol, Salzsäure neutralisiert.

Die Reaktion zwischen Calciumcarbonat und Salzsäure ist die folgende (wenn man annimmt, dass die Kohlensäure sofort zerfällt, was sie relativ schnell auch tut):
\begin{center}
    \ce{2HCl + CaCO3 <=> CaCl2 + CO2 + H2O}
\end{center}

Man braucht also 2 Mol Salzsäure, um 1 Mol Calciumcarbonat zu neutralisieren. Somit ist in der Tablette 0.003415 Mol Calciumcarbonat enthalten.
Da es bei molaren Mengenangaben um die Menge an Teilchen geht, entspricht das auch 0.003415 Mol Calcium pro Tablette.
Die molare Masse von Calcium ist 40.078g pro Mol.
Somit ist in der Tablette 0.13686637g Calcium enthalten, also 136.9mg.

Da es bei einer Titration mit so kleinen Werten und dem Vorgehen einer Rücktitration zu kleinen Verlusten und Abweichungen kommt, klingen 136.9mg plausibel genug um sagen zu können, dass der Hersteller die Wahrheit sagt.

\end{document}
