\documentclass[12pt, letterpaper]{article}
\usepackage[ngerman]{babel}
\usepackage{graphicx}
\usepackage[utf8]{inputenc}
\usepackage{gensymb}
\usepackage{blindtext}
\usepackage{geometry}
\usepackage{url}
\usepackage{seqsplit}
\usepackage{hyperref}
\usepackage{xcolor}

\graphicspath{{images/}}
\setlength{\parindent}{0cm}
\bibliographystyle{gerplain}
\geometry{letterpaper, margin=1in}

\definecolor{color1}{HTML}{0ff500}
\definecolor{color2}{HTML}{64de0b}
\definecolor{color3}{HTML}{9a9a9a}
\definecolor{color4}{HTML}{b5b5b5}
\definecolor{color5}{HTML}{dbdbdb}

\begin{document}

\begin{center}
  \makebox[\textwidth]{\includegraphics[height=\textheight]{"gul"}}
\end{center}

\section{Farbpalette}
\colorbox{color1}{\#0ff500}
\colorbox{color2}{\#63de0b}
\colorbox{color3}{\#9a9a9a}
\colorbox{color4}{\#b5b5b5}
\colorbox{color5}{\#dbdbdb}

\section{Begründung}
Zu sehen ist ein Cover des Magazins "G+L - Magazin für Landschaftsarchitektur und Stadtplanung". Im Zentrum des Covers befindet sich eine Pflanze und der Titel der Ausgabe: "Nachwuchs". Darüber befindet sich das Logo von G+L. Der Hintergrund besteht aus verschiedenen Grautönen.
Die Pflanze besteht aus einem gesättigten grellen Grün in den Blättern und einem eher matten und unauffälligen Grün im Stängel, was einen Qualitätskontrast darstellt, obwohl die Farben auch in gewisser Weise hamonieren, da sie sehr nah aneinander liegen. Das Logo ist in der gleichen Farbe wie die Blätter, sticht aber nicht so heraus, da der Hintergund heller ist als hinter der Pflanze, wodurch der Simultankontrast nicht so stark zur Geltung kommt. Der Grautöne des Hintergrunds sind, wie der Name es schon sagt, eher im Hintergrund und heben somit die Pflanze und das Logo hervor, wodurch sich ein Unbuntkontrast bildet.
Durch dieses Hervorheben der Pflanze, ist sie das Erste, was man sieht, wenn man auf das Cover blickt. Aufgrund des Kontrastes mit dem Hintergrund strahlen die Blätter sehr. Der graue Hintergrund soll gleichzeitig einen Felsen symbolisieren, auf dem die Pflanze als einzige wächst und sich durchsetzt, was man auch an ihrer, durch die Sättigung symbolisierten, Stärke sehen kann. Das passt zu dem Thema Nachwuchs, da durch die Entwicklung des Klimawandels nur sehr beständige Pflanzen weiter bestehen werden und diese sich entsprechend an neue Gegebenheiten adaptieren müssen. Als nächstes betrachtet man das Logo, welches, im Gegensatz zu jeglichem anderen Text, auch die grelle Farbe teilt. Der restliche Text ist in der stumpferen Farbe, damit er nicht die initiale Aufmerksamkeit stielt, sondern erst relevant wird, wenn der potentielle Käufer bereits interessiert ist.

\end{document}
